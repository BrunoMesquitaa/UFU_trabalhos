% Options for packages loaded elsewhere
\PassOptionsToPackage{unicode}{hyperref}
\PassOptionsToPackage{hyphens}{url}
%
\documentclass[
]{article}
\usepackage{amsmath,amssymb}
\usepackage{iftex}
\ifPDFTeX
  \usepackage[T1]{fontenc}
  \usepackage[utf8]{inputenc}
  \usepackage{textcomp} % provide euro and other symbols
\else % if luatex or xetex
  \usepackage{unicode-math} % this also loads fontspec
  \defaultfontfeatures{Scale=MatchLowercase}
  \defaultfontfeatures[\rmfamily]{Ligatures=TeX,Scale=1}
\fi
\usepackage{lmodern}
\ifPDFTeX\else
  % xetex/luatex font selection
\fi
% Use upquote if available, for straight quotes in verbatim environments
\IfFileExists{upquote.sty}{\usepackage{upquote}}{}
\IfFileExists{microtype.sty}{% use microtype if available
  \usepackage[]{microtype}
  \UseMicrotypeSet[protrusion]{basicmath} % disable protrusion for tt fonts
}{}
\makeatletter
\@ifundefined{KOMAClassName}{% if non-KOMA class
  \IfFileExists{parskip.sty}{%
    \usepackage{parskip}
  }{% else
    \setlength{\parindent}{0pt}
    \setlength{\parskip}{6pt plus 2pt minus 1pt}}
}{% if KOMA class
  \KOMAoptions{parskip=half}}
\makeatother
\usepackage{xcolor}
\usepackage[margin=1in]{geometry}
\usepackage{color}
\usepackage{fancyvrb}
\newcommand{\VerbBar}{|}
\newcommand{\VERB}{\Verb[commandchars=\\\{\}]}
\DefineVerbatimEnvironment{Highlighting}{Verbatim}{commandchars=\\\{\}}
% Add ',fontsize=\small' for more characters per line
\usepackage{framed}
\definecolor{shadecolor}{RGB}{248,248,248}
\newenvironment{Shaded}{\begin{snugshade}}{\end{snugshade}}
\newcommand{\AlertTok}[1]{\textcolor[rgb]{0.94,0.16,0.16}{#1}}
\newcommand{\AnnotationTok}[1]{\textcolor[rgb]{0.56,0.35,0.01}{\textbf{\textit{#1}}}}
\newcommand{\AttributeTok}[1]{\textcolor[rgb]{0.13,0.29,0.53}{#1}}
\newcommand{\BaseNTok}[1]{\textcolor[rgb]{0.00,0.00,0.81}{#1}}
\newcommand{\BuiltInTok}[1]{#1}
\newcommand{\CharTok}[1]{\textcolor[rgb]{0.31,0.60,0.02}{#1}}
\newcommand{\CommentTok}[1]{\textcolor[rgb]{0.56,0.35,0.01}{\textit{#1}}}
\newcommand{\CommentVarTok}[1]{\textcolor[rgb]{0.56,0.35,0.01}{\textbf{\textit{#1}}}}
\newcommand{\ConstantTok}[1]{\textcolor[rgb]{0.56,0.35,0.01}{#1}}
\newcommand{\ControlFlowTok}[1]{\textcolor[rgb]{0.13,0.29,0.53}{\textbf{#1}}}
\newcommand{\DataTypeTok}[1]{\textcolor[rgb]{0.13,0.29,0.53}{#1}}
\newcommand{\DecValTok}[1]{\textcolor[rgb]{0.00,0.00,0.81}{#1}}
\newcommand{\DocumentationTok}[1]{\textcolor[rgb]{0.56,0.35,0.01}{\textbf{\textit{#1}}}}
\newcommand{\ErrorTok}[1]{\textcolor[rgb]{0.64,0.00,0.00}{\textbf{#1}}}
\newcommand{\ExtensionTok}[1]{#1}
\newcommand{\FloatTok}[1]{\textcolor[rgb]{0.00,0.00,0.81}{#1}}
\newcommand{\FunctionTok}[1]{\textcolor[rgb]{0.13,0.29,0.53}{\textbf{#1}}}
\newcommand{\ImportTok}[1]{#1}
\newcommand{\InformationTok}[1]{\textcolor[rgb]{0.56,0.35,0.01}{\textbf{\textit{#1}}}}
\newcommand{\KeywordTok}[1]{\textcolor[rgb]{0.13,0.29,0.53}{\textbf{#1}}}
\newcommand{\NormalTok}[1]{#1}
\newcommand{\OperatorTok}[1]{\textcolor[rgb]{0.81,0.36,0.00}{\textbf{#1}}}
\newcommand{\OtherTok}[1]{\textcolor[rgb]{0.56,0.35,0.01}{#1}}
\newcommand{\PreprocessorTok}[1]{\textcolor[rgb]{0.56,0.35,0.01}{\textit{#1}}}
\newcommand{\RegionMarkerTok}[1]{#1}
\newcommand{\SpecialCharTok}[1]{\textcolor[rgb]{0.81,0.36,0.00}{\textbf{#1}}}
\newcommand{\SpecialStringTok}[1]{\textcolor[rgb]{0.31,0.60,0.02}{#1}}
\newcommand{\StringTok}[1]{\textcolor[rgb]{0.31,0.60,0.02}{#1}}
\newcommand{\VariableTok}[1]{\textcolor[rgb]{0.00,0.00,0.00}{#1}}
\newcommand{\VerbatimStringTok}[1]{\textcolor[rgb]{0.31,0.60,0.02}{#1}}
\newcommand{\WarningTok}[1]{\textcolor[rgb]{0.56,0.35,0.01}{\textbf{\textit{#1}}}}
\usepackage{graphicx}
\makeatletter
\def\maxwidth{\ifdim\Gin@nat@width>\linewidth\linewidth\else\Gin@nat@width\fi}
\def\maxheight{\ifdim\Gin@nat@height>\textheight\textheight\else\Gin@nat@height\fi}
\makeatother
% Scale images if necessary, so that they will not overflow the page
% margins by default, and it is still possible to overwrite the defaults
% using explicit options in \includegraphics[width, height, ...]{}
\setkeys{Gin}{width=\maxwidth,height=\maxheight,keepaspectratio}
% Set default figure placement to htbp
\makeatletter
\def\fps@figure{htbp}
\makeatother
\setlength{\emergencystretch}{3em} % prevent overfull lines
\providecommand{\tightlist}{%
  \setlength{\itemsep}{0pt}\setlength{\parskip}{0pt}}
\setcounter{secnumdepth}{-\maxdimen} % remove section numbering
\ifLuaTeX
  \usepackage{selnolig}  % disable illegal ligatures
\fi
\IfFileExists{bookmark.sty}{\usepackage{bookmark}}{\usepackage{hyperref}}
\IfFileExists{xurl.sty}{\usepackage{xurl}}{} % add URL line breaks if available
\urlstyle{same}
\hypersetup{
  pdftitle={Ultimo Trabalho Regressao},
  pdfauthor={Bruno Mesquita dos Santos},
  hidelinks,
  pdfcreator={LaTeX via pandoc}}

\title{Ultimo Trabalho Regressao}
\author{Bruno Mesquita dos Santos}
\date{2023-11-15}

\begin{document}
\maketitle

\hypertarget{biblioteca}{%
\section{Biblioteca}\label{biblioteca}}

\begin{Shaded}
\begin{Highlighting}[]
\ControlFlowTok{if}\NormalTok{ (}\SpecialCharTok{!}\FunctionTok{require}\NormalTok{(}\StringTok{\textquotesingle{}readr\textquotesingle{}}\NormalTok{))}\FunctionTok{install.packages}\NormalTok{(}\StringTok{"readr"}\NormalTok{);}\FunctionTok{library}\NormalTok{(readr)}
\end{Highlighting}
\end{Shaded}

\begin{verbatim}
## Loading required package: readr
\end{verbatim}

\begin{Shaded}
\begin{Highlighting}[]
\ControlFlowTok{if}\NormalTok{ (}\SpecialCharTok{!}\FunctionTok{require}\NormalTok{(}\StringTok{\textquotesingle{}faraway\textquotesingle{}}\NormalTok{))}\FunctionTok{install.packages}\NormalTok{(}\StringTok{"faraway"}\NormalTok{);}\FunctionTok{library}\NormalTok{(faraway)}
\end{Highlighting}
\end{Shaded}

\begin{verbatim}
## Loading required package: faraway
\end{verbatim}

\begin{Shaded}
\begin{Highlighting}[]
\ControlFlowTok{if}\NormalTok{ (}\SpecialCharTok{!}\FunctionTok{require}\NormalTok{(}\StringTok{\textquotesingle{}car\textquotesingle{}}\NormalTok{))}\FunctionTok{install.packages}\NormalTok{(}\StringTok{"car"}\NormalTok{);}\FunctionTok{library}\NormalTok{(car)}
\end{Highlighting}
\end{Shaded}

\begin{verbatim}
## Loading required package: car
\end{verbatim}

\begin{verbatim}
## Loading required package: carData
\end{verbatim}

\begin{verbatim}
## 
## Attaching package: 'car'
\end{verbatim}

\begin{verbatim}
## The following objects are masked from 'package:faraway':
## 
##     logit, vif
\end{verbatim}

\begin{Shaded}
\begin{Highlighting}[]
\ControlFlowTok{if}\NormalTok{ (}\SpecialCharTok{!}\FunctionTok{require}\NormalTok{(}\StringTok{\textquotesingle{}olsrr\textquotesingle{}}\NormalTok{))}\FunctionTok{install.packages}\NormalTok{(}\StringTok{"olsrr"}\NormalTok{);}\FunctionTok{library}\NormalTok{(olsrr)}
\end{Highlighting}
\end{Shaded}

\begin{verbatim}
## Loading required package: olsrr
\end{verbatim}

\begin{verbatim}
## 
## Attaching package: 'olsrr'
\end{verbatim}

\begin{verbatim}
## The following object is masked from 'package:faraway':
## 
##     hsb
\end{verbatim}

\begin{verbatim}
## The following object is masked from 'package:datasets':
## 
##     rivers
\end{verbatim}

\begin{Shaded}
\begin{Highlighting}[]
\ControlFlowTok{if}\NormalTok{ (}\SpecialCharTok{!}\FunctionTok{require}\NormalTok{(}\StringTok{\textquotesingle{}DescTools\textquotesingle{}}\NormalTok{))}\FunctionTok{install.packages}\NormalTok{(}\StringTok{"DescTools"}\NormalTok{);}\FunctionTok{library}\NormalTok{(DescTools)}
\end{Highlighting}
\end{Shaded}

\begin{verbatim}
## Loading required package: DescTools
\end{verbatim}

\begin{verbatim}
## 
## Attaching package: 'DescTools'
\end{verbatim}

\begin{verbatim}
## The following object is masked from 'package:car':
## 
##     Recode
\end{verbatim}

\hypertarget{dados-consumo-cerveja}{%
\section{Dados Consumo Cerveja}\label{dados-consumo-cerveja}}

\begin{Shaded}
\begin{Highlighting}[]
\NormalTok{Consumo\_cerveja }\OtherTok{\textless{}{-}} \FunctionTok{read\_csv}\NormalTok{(}\StringTok{"dados/Consumo\_cerveja.csv"}\NormalTok{,   }
                            \AttributeTok{locale =} \FunctionTok{locale}\NormalTok{(}\AttributeTok{decimal\_mark =} \StringTok{","}\NormalTok{))}
\end{Highlighting}
\end{Shaded}

\begin{verbatim}
## Rows: 941 Columns: 7
## -- Column specification --------------------------------------------------------
## Delimiter: ","
## dbl  (5): Temperatura Media (C), Temperatura Minima (C), Temperatura Maxima ...
## date (1): Data
## 
## i Use `spec()` to retrieve the full column specification for this data.
## i Specify the column types or set `show_col_types = FALSE` to quiet this message.
\end{verbatim}

\hypertarget{funuxe7uxf5es}{%
\section{Funções}\label{funuxe7uxf5es}}

\begin{Shaded}
\begin{Highlighting}[]
\NormalTok{rm\_accent }\OtherTok{\textless{}{-}} \ControlFlowTok{function}\NormalTok{(x) }\FunctionTok{iconv}\NormalTok{(x, }\AttributeTok{to =} \StringTok{"ASCII//TRANSLIT"}\NormalTok{)}
\end{Highlighting}
\end{Shaded}

\hypertarget{limpeza}{%
\section{Limpeza}\label{limpeza}}

\begin{Shaded}
\begin{Highlighting}[]
\FunctionTok{names}\NormalTok{(Consumo\_cerveja) }\OtherTok{\textless{}{-}} \FunctionTok{tolower}\NormalTok{(}
  \FunctionTok{gsub}\NormalTok{(}\StringTok{\textquotesingle{} \textquotesingle{}}\NormalTok{,}\StringTok{\textquotesingle{}\_\textquotesingle{}}\NormalTok{,}\FunctionTok{gsub}\NormalTok{(}\StringTok{\textquotesingle{}[)]\textquotesingle{}}\NormalTok{,}\StringTok{\textquotesingle{}\textquotesingle{}}\NormalTok{,}\FunctionTok{gsub}\NormalTok{(}\StringTok{\textquotesingle{}[(]\textquotesingle{}}\NormalTok{,}\StringTok{\textquotesingle{}\textquotesingle{}}\NormalTok{,}\FunctionTok{rm\_accent}\NormalTok{(}\FunctionTok{names}\NormalTok{(Consumo\_cerveja)))))}
\NormalTok{  )}
\NormalTok{Consumo\_cerveja }\OtherTok{\textless{}{-}} \FunctionTok{na.omit}\NormalTok{(Consumo\_cerveja)}
\end{Highlighting}
\end{Shaded}

\hypertarget{separar-bases}{%
\section{Separar Bases}\label{separar-bases}}

\begin{Shaded}
\begin{Highlighting}[]
\NormalTok{data }\OtherTok{\textless{}{-}}\NormalTok{ Consumo\_cerveja}\SpecialCharTok{$}\NormalTok{data}
\NormalTok{temperatura\_media\_c }\OtherTok{\textless{}{-}}\NormalTok{ Consumo\_cerveja}\SpecialCharTok{$}\NormalTok{temperatura\_media\_c}
\NormalTok{temperatura\_minima\_c }\OtherTok{\textless{}{-}}\NormalTok{ Consumo\_cerveja}\SpecialCharTok{$}\NormalTok{temperatura\_minima\_c}
\NormalTok{temperatura\_maxima\_c }\OtherTok{\textless{}{-}}\NormalTok{ Consumo\_cerveja}\SpecialCharTok{$}\NormalTok{temperatura\_maxima\_c}
\NormalTok{precipitacao\_mm }\OtherTok{\textless{}{-}}\NormalTok{ Consumo\_cerveja}\SpecialCharTok{$}\NormalTok{precipitacao\_mm}
\NormalTok{final\_de\_semana }\OtherTok{\textless{}{-}}\NormalTok{ Consumo\_cerveja}\SpecialCharTok{$}\NormalTok{final\_de\_semana}
\NormalTok{consumo\_de\_cerveja\_litros }\OtherTok{\textless{}{-}}\NormalTok{ Consumo\_cerveja}\SpecialCharTok{$}\NormalTok{consumo\_de\_cerveja\_litros}
\end{Highlighting}
\end{Shaded}

\hypertarget{anuxe1lise-de-regressuxe3o}{%
\section{Análise de Regressão}\label{anuxe1lise-de-regressuxe3o}}

\hypertarget{questuxe3o-a}{%
\subsection{Questão A}\label{questuxe3o-a}}

Ajuste um modelo de regressão linear múltipla considerando todas as
variáveis independentes. Verifique a multicolinearidade entre as
variáveis independentes, e se há necessidade de excluir alguma delas por
esse critério. Em caso afirmativo, ajuste novo modelo sem essa variável.
Apresente todos os valores de Vif.

\begin{Shaded}
\begin{Highlighting}[]
\NormalTok{modelo\_completo }\OtherTok{=} \FunctionTok{lm}\NormalTok{(consumo\_de\_cerveja\_litros }\SpecialCharTok{\textasciitilde{}}\NormalTok{ data}\SpecialCharTok{+}\NormalTok{temperatura\_media\_c}\SpecialCharTok{+}
\NormalTok{                       temperatura\_minima\_c}\SpecialCharTok{+}\NormalTok{temperatura\_maxima\_c}\SpecialCharTok{+}
\NormalTok{                       precipitacao\_mm}\SpecialCharTok{+}\NormalTok{final\_de\_semana)}

\FunctionTok{summary}\NormalTok{(modelo\_completo)}
\end{Highlighting}
\end{Shaded}

\begin{verbatim}
## 
## Call:
## lm(formula = consumo_de_cerveja_litros ~ data + temperatura_media_c + 
##     temperatura_minima_c + temperatura_maxima_c + precipitacao_mm + 
##     final_de_semana)
## 
## Residuals:
##     Min      1Q  Median      3Q     Max 
## -4823.8 -1794.4  -193.9  1804.6  5514.2 
## 
## Coefficients:
##                        Estimate Std. Error t value Pr(>|t|)    
## (Intercept)          -60845.145  19516.939  -3.118 0.001971 ** 
## data                      4.027      1.167   3.451 0.000625 ***
## temperatura_media_c     -30.772    186.065  -0.165 0.868737    
## temperatura_minima_c     46.816    110.380   0.424 0.671725    
## temperatura_maxima_c    675.535     93.905   7.194 3.72e-12 ***
## precipitacao_mm         -58.485      9.892  -5.913 7.86e-09 ***
## final_de_semana        5198.875    267.020  19.470  < 2e-16 ***
## ---
## Signif. codes:  0 '***' 0.001 '**' 0.01 '*' 0.05 '.' 0.1 ' ' 1
## 
## Residual standard error: 2298 on 358 degrees of freedom
## Multiple R-squared:  0.7316, Adjusted R-squared:  0.7271 
## F-statistic: 162.6 on 6 and 358 DF,  p-value: < 2.2e-16
\end{verbatim}

\begin{Shaded}
\begin{Highlighting}[]
\FunctionTok{summary}\NormalTok{(}\FunctionTok{aov}\NormalTok{(modelo\_completo))}
\end{Highlighting}
\end{Shaded}

\begin{verbatim}
##                       Df    Sum Sq   Mean Sq F value   Pr(>F)    
## data                   1 1.335e+07 1.335e+07   2.528    0.113    
## temperatura_media_c    1 2.384e+09 2.384e+09 451.367  < 2e-16 ***
## temperatura_minima_c   1 2.540e+08 2.540e+08  48.098 1.90e-11 ***
## temperatura_maxima_c   1 3.247e+08 3.247e+08  61.481 5.18e-14 ***
## precipitacao_mm        1 1.752e+08 1.752e+08  33.169 1.82e-08 ***
## final_de_semana        1 2.002e+09 2.002e+09 379.081  < 2e-16 ***
## Residuals            358 1.891e+09 5.282e+06                     
## ---
## Signif. codes:  0 '***' 0.001 '**' 0.01 '*' 0.05 '.' 0.1 ' ' 1
\end{verbatim}

\hypertarget{anotacuxe3o}{%
\subsubsection{Anotacão:}\label{anotacuxe3o}}

Se o VIF for igual a 1 não há multicolinearidade entre os fatores, mas
se o VIF for maior que 1, as preditoras podem estar moderadamente
correlacionadas. A saída acima mostra que o VIF para os fatores de
publicação e anos são cerca de 1.5, o que indica alguma correlação, mas
não o suficiente para se preocupar demais com isso. Um VIF entre 5 e 10
indica alta correlação, o que pode ser problemático. E se o VIF for
acima de 10, você pode assumir que os coeficientes de regressão estão
mal estimados devido à multicolinearidade.

\hypertarget{vif}{%
\subsubsection{VIF:}\label{vif}}

\begin{Shaded}
\begin{Highlighting}[]
\NormalTok{vif }\OtherTok{\textless{}{-}} \DecValTok{1}\SpecialCharTok{/}\NormalTok{(}\DecValTok{1}\SpecialCharTok{{-}}\FunctionTok{summary}\NormalTok{(modelo\_completo)[[}\StringTok{"r.squared"}\NormalTok{]])}
\FunctionTok{paste}\NormalTok{(}\StringTok{\textquotesingle{}vif =\textquotesingle{}}\NormalTok{,vif)}
\end{Highlighting}
\end{Shaded}

\begin{verbatim}
## [1] "vif = 3.72548692094354"
\end{verbatim}

Com relação ao nosso modelo temos um VIF para o modelo completo de
3.7254 o que quer dizer que ele é moderadamente correlacionadas.

\begin{Shaded}
\begin{Highlighting}[]
\FunctionTok{vif}\NormalTok{(modelo\_completo)}
\end{Highlighting}
\end{Shaded}

\begin{verbatim}
##                 data  temperatura_media_c temperatura_minima_c 
##             1.044990            24.129091             6.706766 
## temperatura_maxima_c      precipitacao_mm      final_de_semana 
##            11.327898             1.039824             1.003914
\end{verbatim}

Porém quando vamos para um VIF mais detalhado temos em
temperatura\_media\_c, temperatura\_minima\_c, temperatura\_maxima\_c
uma multicolinearidade alto, o que faz sentido devido serem dados de
temperaturas tendo a mesmas representatividade, logo removeria dois
ficando é com a temperatura\_media\_c.

\hypertarget{removendo-a-multicolinearidade}{%
\subsubsection{Removendo a
multicolinearidade}\label{removendo-a-multicolinearidade}}

\begin{Shaded}
\begin{Highlighting}[]
\NormalTok{modelo\_completo }\OtherTok{=} \FunctionTok{lm}\NormalTok{(consumo\_de\_cerveja\_litros }\SpecialCharTok{\textasciitilde{}}\NormalTok{ data}\SpecialCharTok{+}\NormalTok{temperatura\_media\_c}\SpecialCharTok{+}
\NormalTok{                       precipitacao\_mm}\SpecialCharTok{+}\NormalTok{final\_de\_semana)}

\FunctionTok{summary}\NormalTok{(modelo\_completo)}
\end{Highlighting}
\end{Shaded}

\begin{verbatim}
## 
## Call:
## lm(formula = consumo_de_cerveja_litros ~ data + temperatura_media_c + 
##     precipitacao_mm + final_de_semana)
## 
## Residuals:
##     Min      1Q  Median      3Q     Max 
## -6190.9 -1911.3  -251.5  1995.6  6504.4 
## 
## Coefficients:
##                       Estimate Std. Error t value Pr(>|t|)    
## (Intercept)         -69560.767  21094.827  -3.298 0.001073 ** 
## data                     4.556      1.263   3.608 0.000353 ***
## temperatura_media_c    854.735     41.965  20.368  < 2e-16 ***
## precipitacao_mm        -74.587     10.681  -6.983  1.4e-11 ***
## final_de_semana       5239.630    293.724  17.839  < 2e-16 ***
## ---
## Signif. codes:  0 '***' 0.001 '**' 0.01 '*' 0.05 '.' 0.1 ' ' 1
## 
## Residual standard error: 2530 on 360 degrees of freedom
## Multiple R-squared:  0.673,  Adjusted R-squared:  0.6694 
## F-statistic: 185.2 on 4 and 360 DF,  p-value: < 2.2e-16
\end{verbatim}

\begin{Shaded}
\begin{Highlighting}[]
\FunctionTok{summary}\NormalTok{(}\FunctionTok{aov}\NormalTok{(modelo\_completo))}
\end{Highlighting}
\end{Shaded}

\begin{verbatim}
##                      Df    Sum Sq   Mean Sq F value   Pr(>F)    
## data                  1 1.335e+07 1.335e+07   2.087    0.149    
## temperatura_media_c   1 2.384e+09 2.384e+09 372.581  < 2e-16 ***
## precipitacao_mm       1 3.074e+08 3.074e+08  48.038 1.94e-11 ***
## final_de_semana       1 2.036e+09 2.036e+09 318.216  < 2e-16 ***
## Residuals           360 2.303e+09 6.399e+06                     
## ---
## Signif. codes:  0 '***' 0.001 '**' 0.01 '*' 0.05 '.' 0.1 ' ' 1
\end{verbatim}

\hypertarget{conclusuxe3o}{%
\subsubsection{Conclusão:}\label{conclusuxe3o}}

\begin{Shaded}
\begin{Highlighting}[]
\NormalTok{vif }\OtherTok{\textless{}{-}} \DecValTok{1}\SpecialCharTok{/}\NormalTok{(}\DecValTok{1}\SpecialCharTok{{-}}\FunctionTok{summary}\NormalTok{(modelo\_completo)[[}\StringTok{"r.squared"}\NormalTok{]])}
\FunctionTok{paste}\NormalTok{(}\StringTok{\textquotesingle{}vif =\textquotesingle{}}\NormalTok{,vif)}
\end{Highlighting}
\end{Shaded}

\begin{verbatim}
## [1] "vif = 3.05811655234158"
\end{verbatim}

Com relação ao nosso modelo temos um VIF para o modelo completo de 3.058
o que quer dizer que ele é moderadamente correlacionadas.

\begin{Shaded}
\begin{Highlighting}[]
\FunctionTok{vif}\NormalTok{(modelo\_completo)}
\end{Highlighting}
\end{Shaded}

\begin{verbatim}
##                data temperatura_media_c     precipitacao_mm     final_de_semana 
##            1.010045            1.013175            1.000704            1.002719
\end{verbatim}

Porém quando vamos para um VIF mais detalhado temos que não há
multicolinearidade entre os fatores.

\hypertarget{questuxe3o-b}{%
\subsection{Questão B}\label{questuxe3o-b}}

Escreva as hipóteses, decisão e conclusão do teste F para o modelo. Use
o pvalor da saída do software para o teste. Faça a interpretação do
coeficiente de determinação.

\begin{Shaded}
\begin{Highlighting}[]
\FunctionTok{summary}\NormalTok{(modelo\_completo)}
\end{Highlighting}
\end{Shaded}

\begin{verbatim}
## 
## Call:
## lm(formula = consumo_de_cerveja_litros ~ data + temperatura_media_c + 
##     precipitacao_mm + final_de_semana)
## 
## Residuals:
##     Min      1Q  Median      3Q     Max 
## -6190.9 -1911.3  -251.5  1995.6  6504.4 
## 
## Coefficients:
##                       Estimate Std. Error t value Pr(>|t|)    
## (Intercept)         -69560.767  21094.827  -3.298 0.001073 ** 
## data                     4.556      1.263   3.608 0.000353 ***
## temperatura_media_c    854.735     41.965  20.368  < 2e-16 ***
## precipitacao_mm        -74.587     10.681  -6.983  1.4e-11 ***
## final_de_semana       5239.630    293.724  17.839  < 2e-16 ***
## ---
## Signif. codes:  0 '***' 0.001 '**' 0.01 '*' 0.05 '.' 0.1 ' ' 1
## 
## Residual standard error: 2530 on 360 degrees of freedom
## Multiple R-squared:  0.673,  Adjusted R-squared:  0.6694 
## F-statistic: 185.2 on 4 and 360 DF,  p-value: < 2.2e-16
\end{verbatim}

\hypertarget{hipuxf3tese}{%
\subsubsection{Hipótese:}\label{hipuxf3tese}}

\[
\left\{ \begin{array}{rc} 
H0: \beta1 = \beta2 = \beta3 = \beta4 = 0 \\ 
H1: pelo \ menos \ um \ \beta \ difere \ de \ zero \\ 
\end{array}\right.
\]

\hypertarget{conclusuxe3o-1}{%
\subsubsection{Conclusão}\label{conclusuxe3o-1}}

\begin{itemize}
\tightlist
\item
  Dado que nosso modelo completo possuí p-value: \textless{} 2.2e-16,
  podemos concluir que há qualquer nível de significância ele rejeita
  H0, logo pelo menos um beta difere de zero (2.2e-16 \textless{}
  0.001). Temos uma F calc de 185.2 o que é bem expressivo e positivo.
\item
  Sendo R-squared = 0.673 e Adjusted R-squared = 0.6694, o que
  relativamente aceitavel já que identifica a porcentagem a
  variabilidade no campo Y (consumo\_de\_cerveja\_litros) que é
  explicada pela variaveis regressoras.
\end{itemize}

\hypertarget{questuxe3o-c}{%
\subsection{Questão C}\label{questuxe3o-c}}

Escreva as hipóteses, decisão e conclusão do teste t para todos os
parâmetros do modelo. Decida quais variáveis não são importantes neste
modelo e porque. Use 5\% de significância, e considere a regra do pvalor
para decisão.

\hypertarget{hipuxf3tese-1}{%
\subsubsection{Hipótese 1:}\label{hipuxf3tese-1}}

\[
\left\{ \begin{array}{rc} 
H0: \beta1 = 0 \\ 
H1: \beta1 \neq 0 \\ 
\end{array}\right.
\]

\begin{Shaded}
\begin{Highlighting}[]
\NormalTok{modelo\_uma\_var }\OtherTok{=} \FunctionTok{lm}\NormalTok{(consumo\_de\_cerveja\_litros }\SpecialCharTok{\textasciitilde{}}\NormalTok{ data)}
\FunctionTok{summary}\NormalTok{(modelo\_uma\_var)}
\end{Highlighting}
\end{Shaded}

\begin{verbatim}
## 
## Call:
## lm(formula = consumo_de_cerveja_litros ~ data)
## 
## Residuals:
##      Min       1Q   Median       3Q      Max 
## -11094.7  -3396.6   -394.7   3309.0  12849.7 
## 
## Coefficients:
##              Estimate Std. Error t value Pr(>|t|)
## (Intercept) -4766.744  36332.377  -0.131    0.896
## data            1.815      2.186   0.830    0.407
## 
## Residual standard error: 4401 on 363 degrees of freedom
## Multiple R-squared:  0.001896,   Adjusted R-squared:  -0.0008538 
## F-statistic: 0.6895 on 1 and 363 DF,  p-value: 0.4069
\end{verbatim}

Concluimos a um nível de 5\% de significância que aceita-se H0 dado que
p-value = 0.4069 é maior que o nível significância de 0.05. Logo o
coeficiente angular 𝛽1 é estatisticamente igual a 0, desta forma a
variável data não serve para predizer os valores de
y(consumo\_de\_cerveja\_litros).

\hypertarget{hipuxf3tese-2}{%
\subsubsection{Hipótese 2:}\label{hipuxf3tese-2}}

\[
\left\{ \begin{array}{rc} 
H0: \beta2 = 0 \\ 
H1: \beta2 \neq 0 \\ 
\end{array}\right.
\]

\begin{Shaded}
\begin{Highlighting}[]
\NormalTok{modelo\_uma\_var }\OtherTok{=} \FunctionTok{lm}\NormalTok{(consumo\_de\_cerveja\_litros }\SpecialCharTok{\textasciitilde{}}\NormalTok{ temperatura\_media\_c)}
\FunctionTok{summary}\NormalTok{(modelo\_uma\_var)}
\end{Highlighting}
\end{Shaded}

\begin{verbatim}
## 
## Call:
## lm(formula = consumo_de_cerveja_litros ~ temperatura_media_c)
## 
## Residuals:
##     Min      1Q  Median      3Q     Max 
## -9221.4 -2845.5  -315.3  2409.0  9392.5 
## 
## Coefficients:
##                     Estimate Std. Error t value Pr(>|t|)    
## (Intercept)          8528.91    1275.36   6.687  8.6e-11 ***
## temperatura_media_c   794.88      59.42  13.377  < 2e-16 ***
## ---
## Signif. codes:  0 '***' 0.001 '**' 0.01 '*' 0.05 '.' 0.1 ' ' 1
## 
## Residual standard error: 3605 on 363 degrees of freedom
## Multiple R-squared:  0.3302, Adjusted R-squared:  0.3283 
## F-statistic: 178.9 on 1 and 363 DF,  p-value: < 2.2e-16
\end{verbatim}

Concluimos a um nível de 5\% de significância que rejeita-se H0 dado que
p-value = 2.2e-16 é menor que o nível significância de 0.05. Logo o
coeficiente angular 𝛽2 é estatisticamente diferente de 0, desta forma a
variável temperatura\_media\_c é útil para predizer os valores de
y(consumo\_de\_cerveja\_litros).

\hypertarget{hipuxf3tese-3}{%
\subsubsection{Hipótese 3:}\label{hipuxf3tese-3}}

\[
\left\{ \begin{array}{rc} 
H0: \beta3 = 0 \\ 
H1: \beta3 \neq 0 \\ 
\end{array}\right.
\]

\begin{Shaded}
\begin{Highlighting}[]
\NormalTok{modelo\_uma\_var }\OtherTok{=} \FunctionTok{lm}\NormalTok{(consumo\_de\_cerveja\_litros }\SpecialCharTok{\textasciitilde{}}\NormalTok{ precipitacao\_mm)}
\FunctionTok{summary}\NormalTok{(modelo\_uma\_var)}
\end{Highlighting}
\end{Shaded}

\begin{verbatim}
## 
## Call:
## lm(formula = consumo_de_cerveja_litros ~ precipitacao_mm)
## 
## Residuals:
##      Min       1Q   Median       3Q      Max 
## -11415.1  -3252.5   -456.1   3227.2  12178.9 
## 
## Coefficients:
##                 Estimate Std. Error t value Pr(>|t|)    
## (Intercept)     25758.12     245.27 105.021  < 2e-16 ***
## precipitacao_mm   -68.65      18.24  -3.763 0.000195 ***
## ---
## Signif. codes:  0 '***' 0.001 '**' 0.01 '*' 0.05 '.' 0.1 ' ' 1
## 
## Residual standard error: 4322 on 363 degrees of freedom
## Multiple R-squared:  0.03755,    Adjusted R-squared:  0.0349 
## F-statistic: 14.16 on 1 and 363 DF,  p-value: 0.0001954
\end{verbatim}

Concluimos a um nível de 5\% de significância que rejeita-se H0 dado que
p-value = 0.0001954 é menor que o nível significância de 0.05. Logo o
coeficiente angular 𝛽3 é estatisticamente diferente de 0, desta forma a
variável precipitacao\_mm é útil para predizer os valores de
y(consumo\_de\_cerveja\_litros).

\hypertarget{hipuxf3tese-4}{%
\subsubsection{Hipótese 4:}\label{hipuxf3tese-4}}

\[
\left\{ \begin{array}{rc} 
H0: \beta4 = 0 \\ 
H1: \beta4 \neq 0 \\ 
\end{array}\right.
\]

\begin{Shaded}
\begin{Highlighting}[]
\NormalTok{modelo\_uma\_var }\OtherTok{=} \FunctionTok{lm}\NormalTok{(consumo\_de\_cerveja\_litros }\SpecialCharTok{\textasciitilde{}}\NormalTok{ final\_de\_semana)}
\FunctionTok{summary}\NormalTok{(modelo\_uma\_var)}
\end{Highlighting}
\end{Shaded}

\begin{verbatim}
## 
## Call:
## lm(formula = consumo_de_cerveja_litros ~ final_de_semana)
## 
## Residuals:
##     Min      1Q  Median      3Q     Max 
## -9655.2 -2753.2   -61.2  2524.8 11862.8 
## 
## Coefficients:
##                 Estimate Std. Error t value Pr(>|t|)    
## (Intercept)      23998.2      235.2  102.04   <2e-16 ***
## final_de_semana   4924.5      440.6   11.18   <2e-16 ***
## ---
## Signif. codes:  0 '***' 0.001 '**' 0.01 '*' 0.05 '.' 0.1 ' ' 1
## 
## Residual standard error: 3800 on 363 degrees of freedom
## Multiple R-squared:  0.256,  Adjusted R-squared:  0.254 
## F-statistic: 124.9 on 1 and 363 DF,  p-value: < 2.2e-16
\end{verbatim}

Concluimos a um nível de 5\% de significância que rejeita-se H0 dado que
p-value = 2.2e-16 é menor que o nível significância de 0.05. Logo o
coeficiente angular 𝛽4 é estatisticamente diferente de 0, desta forma a
variável final\_de\_semana é útil para predizer os valores de
y(consumo\_de\_cerveja\_litros).

\hypertarget{questuxe3o-d}{%
\subsection{Questão D}\label{questuxe3o-d}}

Utilize o método Backward de seleção de variáveis para encontrar o
melhor conjunto de preditoras para essa variável y. Escreva a equação do
modelo ajustado e a interpretação, para todas as variáveis que restaram
no modelo. Considere 5\% de significância. Apresente os valores dos
testes em cada passo, com a interpretação.

Modelo completo -\textgreater{} consumo\_de\_cerveja\_litros
\textasciitilde{}
data+temperatura\_media\_c+precipitacao\_mm+final\_de\_semana

\hypertarget{removendo-data}{%
\subsubsection{Removendo data}\label{removendo-data}}

\begin{Shaded}
\begin{Highlighting}[]
\NormalTok{modelo\_sem\_data }\OtherTok{=} \FunctionTok{lm}\NormalTok{(consumo\_de\_cerveja\_litros }\SpecialCharTok{\textasciitilde{}} 
\NormalTok{                       temperatura\_media\_c}\SpecialCharTok{+}\NormalTok{precipitacao\_mm}\SpecialCharTok{+}\NormalTok{final\_de\_semana)}
\FunctionTok{anova}\NormalTok{(modelo\_completo,modelo\_sem\_data)}
\end{Highlighting}
\end{Shaded}

\begin{verbatim}
## Analysis of Variance Table
## 
## Model 1: consumo_de_cerveja_litros ~ data + temperatura_media_c + precipitacao_mm + 
##     final_de_semana
## Model 2: consumo_de_cerveja_litros ~ temperatura_media_c + precipitacao_mm + 
##     final_de_semana
##   Res.Df        RSS Df Sum of Sq      F    Pr(>F)    
## 1    360 2303474722                                  
## 2    361 2386755919 -1 -83281197 13.016 0.0003526 ***
## ---
## Signif. codes:  0 '***' 0.001 '**' 0.01 '*' 0.05 '.' 0.1 ' ' 1
\end{verbatim}

\hypertarget{removendo-temperatura_media_c}{%
\subsubsection{Removendo
temperatura\_media\_c}\label{removendo-temperatura_media_c}}

\begin{Shaded}
\begin{Highlighting}[]
\NormalTok{modelo\_sem\_data }\OtherTok{=} \FunctionTok{lm}\NormalTok{(consumo\_de\_cerveja\_litros }\SpecialCharTok{\textasciitilde{}} 
\NormalTok{                       data}\SpecialCharTok{++}\NormalTok{precipitacao\_mm}\SpecialCharTok{+}\NormalTok{final\_de\_semana)}
\FunctionTok{anova}\NormalTok{(modelo\_completo,modelo\_sem\_data)}
\end{Highlighting}
\end{Shaded}

\begin{verbatim}
## Analysis of Variance Table
## 
## Model 1: consumo_de_cerveja_litros ~ data + temperatura_media_c + precipitacao_mm + 
##     final_de_semana
## Model 2: consumo_de_cerveja_litros ~ data + +precipitacao_mm + final_de_semana
##   Res.Df        RSS Df   Sum of Sq      F    Pr(>F)    
## 1    360 2303474722                                    
## 2    361 4957860285 -1 -2654385563 414.84 < 2.2e-16 ***
## ---
## Signif. codes:  0 '***' 0.001 '**' 0.01 '*' 0.05 '.' 0.1 ' ' 1
\end{verbatim}

\hypertarget{removendo-precipitacao_mm}{%
\subsubsection{Removendo
precipitacao\_mm}\label{removendo-precipitacao_mm}}

\begin{Shaded}
\begin{Highlighting}[]
\NormalTok{modelo\_sem\_data }\OtherTok{=} \FunctionTok{lm}\NormalTok{(consumo\_de\_cerveja\_litros }\SpecialCharTok{\textasciitilde{}} 
\NormalTok{                       data}\SpecialCharTok{+}\NormalTok{temperatura\_media\_c}\SpecialCharTok{+}\NormalTok{final\_de\_semana)}
\FunctionTok{anova}\NormalTok{(modelo\_completo,modelo\_sem\_data)}
\end{Highlighting}
\end{Shaded}

\begin{verbatim}
## Analysis of Variance Table
## 
## Model 1: consumo_de_cerveja_litros ~ data + temperatura_media_c + precipitacao_mm + 
##     final_de_semana
## Model 2: consumo_de_cerveja_litros ~ data + temperatura_media_c + final_de_semana
##   Res.Df        RSS Df  Sum of Sq      F    Pr(>F)    
## 1    360 2303474722                                   
## 2    361 2615517078 -1 -312042356 48.768 1.398e-11 ***
## ---
## Signif. codes:  0 '***' 0.001 '**' 0.01 '*' 0.05 '.' 0.1 ' ' 1
\end{verbatim}

\hypertarget{removendo-final_de_semana}{%
\subsubsection{Removendo
final\_de\_semana}\label{removendo-final_de_semana}}

\begin{Shaded}
\begin{Highlighting}[]
\NormalTok{modelo\_sem\_data }\OtherTok{=} \FunctionTok{lm}\NormalTok{(consumo\_de\_cerveja\_litros }\SpecialCharTok{\textasciitilde{}} 
\NormalTok{                       data}\SpecialCharTok{+}\NormalTok{temperatura\_media\_c}\SpecialCharTok{+}\NormalTok{precipitacao\_mm)}
\FunctionTok{anova}\NormalTok{(modelo\_completo,modelo\_sem\_data)}
\end{Highlighting}
\end{Shaded}

\begin{verbatim}
## Analysis of Variance Table
## 
## Model 1: consumo_de_cerveja_litros ~ data + temperatura_media_c + precipitacao_mm + 
##     final_de_semana
## Model 2: consumo_de_cerveja_litros ~ data + temperatura_media_c + precipitacao_mm
##   Res.Df        RSS Df   Sum of Sq      F    Pr(>F)    
## 1    360 2303474722                                    
## 2    361 4339592657 -1 -2036117936 318.22 < 2.2e-16 ***
## ---
## Signif. codes:  0 '***' 0.001 '**' 0.01 '*' 0.05 '.' 0.1 ' ' 1
\end{verbatim}

\begin{Shaded}
\begin{Highlighting}[]
\FunctionTok{qf}\NormalTok{(}\FloatTok{0.95}\NormalTok{,}\DecValTok{1}\NormalTok{,}\DecValTok{361}\NormalTok{)}
\end{Highlighting}
\end{Shaded}

\begin{verbatim}
## [1] 3.867347
\end{verbatim}

\hypertarget{conclusuxe3o-2}{%
\subsubsection{Conclusão:}\label{conclusuxe3o-2}}

Dado que nosso menor F calc foi 13.016, devemos compara com F Tab de
parametro (0.95,1,361). Logo temos que A 95\% de confiança, escolhemos o
Fcalc mínimo para comparar com Ftab, Fmin(13.016) \textgreater{}
Ftab(3.867347), nesse casos rejeita-se Ho, e conclui-se que não se pode
tirar a variável data. Então mantém o modelo com modelo completo.

\hypertarget{questuxe3o-e}{%
\subsection{Questão E}\label{questuxe3o-e}}

Utilize o método Forward de seleção de variáveis para encontrar o melhor
conjunto de preditoras para essa variável y. Escreva a equação do modelo
ajustado e compare com o modelo obtido em (d).

Me basendo no F da questão C o melhor modelo reduzido para se começar é:

\begin{Shaded}
\begin{Highlighting}[]
\NormalTok{modelo\_reduzido }\OtherTok{=} \FunctionTok{lm}\NormalTok{(consumo\_de\_cerveja\_litros }\SpecialCharTok{\textasciitilde{}}\NormalTok{ temperatura\_media\_c)}
\end{Highlighting}
\end{Shaded}

\hypertarget{adicionar-data}{%
\subsubsection{Adicionar data}\label{adicionar-data}}

\begin{Shaded}
\begin{Highlighting}[]
\NormalTok{modelo\_mais\_data }\OtherTok{=} \FunctionTok{lm}\NormalTok{(consumo\_de\_cerveja\_litros }\SpecialCharTok{\textasciitilde{}}\NormalTok{ temperatura\_media\_c}\SpecialCharTok{+}\NormalTok{data)}
\FunctionTok{summary}\NormalTok{(}\FunctionTok{aov}\NormalTok{(modelo\_mais\_data))}
\end{Highlighting}
\end{Shaded}

\begin{verbatim}
##                      Df    Sum Sq   Mean Sq F value Pr(>F)    
## temperatura_media_c   1 2.326e+09 2.326e+09 181.188 <2e-16 ***
## data                  1 7.143e+07 7.143e+07   5.564 0.0189 *  
## Residuals           362 4.647e+09 1.284e+07                   
## ---
## Signif. codes:  0 '***' 0.001 '**' 0.01 '*' 0.05 '.' 0.1 ' ' 1
\end{verbatim}

\begin{Shaded}
\begin{Highlighting}[]
\FunctionTok{anova}\NormalTok{(modelo\_reduzido,modelo\_mais\_data)}
\end{Highlighting}
\end{Shaded}

\begin{verbatim}
## Analysis of Variance Table
## 
## Model 1: consumo_de_cerveja_litros ~ temperatura_media_c
## Model 2: consumo_de_cerveja_litros ~ temperatura_media_c + data
##   Res.Df        RSS Df Sum of Sq      F  Pr(>F)  
## 1    363 4718394688                              
## 2    362 4646965029  1  71429659 5.5644 0.01886 *
## ---
## Signif. codes:  0 '***' 0.001 '**' 0.01 '*' 0.05 '.' 0.1 ' ' 1
\end{verbatim}

\hypertarget{adicionar-precipitacao_mm}{%
\subsubsection{Adicionar
precipitacao\_mm}\label{adicionar-precipitacao_mm}}

\begin{Shaded}
\begin{Highlighting}[]
\NormalTok{modelo\_mais\_precipitacao\_mm }\OtherTok{=} \FunctionTok{lm}\NormalTok{(consumo\_de\_cerveja\_litros }\SpecialCharTok{\textasciitilde{}} 
\NormalTok{                                   temperatura\_media\_c}\SpecialCharTok{+}\NormalTok{precipitacao\_mm)}
\FunctionTok{anova}\NormalTok{(modelo\_reduzido,modelo\_mais\_precipitacao\_mm)}
\end{Highlighting}
\end{Shaded}

\begin{verbatim}
## Analysis of Variance Table
## 
## Model 1: consumo_de_cerveja_litros ~ temperatura_media_c
## Model 2: consumo_de_cerveja_litros ~ temperatura_media_c + precipitacao_mm
##   Res.Df        RSS Df Sum of Sq      F    Pr(>F)    
## 1    363 4718394688                                  
## 2    362 4413993821  1 304400867 24.965 9.107e-07 ***
## ---
## Signif. codes:  0 '***' 0.001 '**' 0.01 '*' 0.05 '.' 0.1 ' ' 1
\end{verbatim}

\hypertarget{adicionar-final_de_semana}{%
\subsubsection{Adicionar
final\_de\_semana}\label{adicionar-final_de_semana}}

\begin{Shaded}
\begin{Highlighting}[]
\NormalTok{modelo\_mais\_final\_de\_semana }\OtherTok{=} \FunctionTok{lm}\NormalTok{(consumo\_de\_cerveja\_litros }\SpecialCharTok{\textasciitilde{}} 
\NormalTok{                                   temperatura\_media\_c}\SpecialCharTok{+}\NormalTok{final\_de\_semana)}
\FunctionTok{anova}\NormalTok{(modelo\_reduzido,modelo\_mais\_final\_de\_semana)}
\end{Highlighting}
\end{Shaded}

\begin{verbatim}
## Analysis of Variance Table
## 
## Model 1: consumo_de_cerveja_litros ~ temperatura_media_c
## Model 2: consumo_de_cerveja_litros ~ temperatura_media_c + final_de_semana
##   Res.Df        RSS Df  Sum of Sq      F    Pr(>F)    
## 1    363 4718394688                                   
## 2    362 2695619426  1 2022775262 271.64 < 2.2e-16 ***
## ---
## Signif. codes:  0 '***' 0.001 '**' 0.01 '*' 0.05 '.' 0.1 ' ' 1
\end{verbatim}

\hypertarget{conclusuxe3o-3}{%
\subsubsection{Conclusão:}\label{conclusuxe3o-3}}

O Fmax é igual 271.64 sendo final\_de\_semana o mais provável de ser
adicionado.

\hypertarget{validando-a-variavel}{%
\subsubsection{Validando a variavel}\label{validando-a-variavel}}

\begin{Shaded}
\begin{Highlighting}[]
\FunctionTok{qf}\NormalTok{(}\FloatTok{0.95}\NormalTok{, }\DecValTok{1}\NormalTok{ , }\DecValTok{362}\NormalTok{)}
\end{Highlighting}
\end{Shaded}

\begin{verbatim}
## [1] 3.867275
\end{verbatim}

\hypertarget{conclusuxe3o-4}{%
\paragraph{Conclusão:}\label{conclusuxe3o-4}}

Fmax(271.64) \textgreater{} Ftab(3.867275) aceitamos o modelo completo
como consumo\_de\_cerveja\_litros \textasciitilde{}
temperatura\_media\_c+final\_de\_semana

\hypertarget{novo-modelo-reduzido}{%
\subsubsection{Novo Modelo Reduzido}\label{novo-modelo-reduzido}}

\begin{Shaded}
\begin{Highlighting}[]
\NormalTok{modelo\_reduzido }\OtherTok{=} \FunctionTok{lm}\NormalTok{(consumo\_de\_cerveja\_litros }\SpecialCharTok{\textasciitilde{}} 
\NormalTok{                       temperatura\_media\_c}\SpecialCharTok{+}\NormalTok{final\_de\_semana)}
\end{Highlighting}
\end{Shaded}

\hypertarget{adicionar-data-1}{%
\subsubsection{Adicionar data}\label{adicionar-data-1}}

\begin{Shaded}
\begin{Highlighting}[]
\NormalTok{modelo\_mais\_data }\OtherTok{=} \FunctionTok{lm}\NormalTok{(consumo\_de\_cerveja\_litros }\SpecialCharTok{\textasciitilde{}} 
\NormalTok{                        temperatura\_media\_c}\SpecialCharTok{+}\NormalTok{final\_de\_semana}\SpecialCharTok{+}\NormalTok{data)}
\FunctionTok{summary}\NormalTok{(}\FunctionTok{aov}\NormalTok{(modelo\_mais\_data))}
\end{Highlighting}
\end{Shaded}

\begin{verbatim}
##                      Df    Sum Sq   Mean Sq F value   Pr(>F)    
## temperatura_media_c   1 2.326e+09 2.326e+09  321.03  < 2e-16 ***
## final_de_semana       1 2.023e+09 2.023e+09  279.19  < 2e-16 ***
## data                  1 8.010e+07 8.010e+07   11.06 0.000975 ***
## Residuals           361 2.616e+09 7.245e+06                     
## ---
## Signif. codes:  0 '***' 0.001 '**' 0.01 '*' 0.05 '.' 0.1 ' ' 1
\end{verbatim}

\begin{Shaded}
\begin{Highlighting}[]
\FunctionTok{anova}\NormalTok{(modelo\_reduzido,modelo\_mais\_data)}
\end{Highlighting}
\end{Shaded}

\begin{verbatim}
## Analysis of Variance Table
## 
## Model 1: consumo_de_cerveja_litros ~ temperatura_media_c + final_de_semana
## Model 2: consumo_de_cerveja_litros ~ temperatura_media_c + final_de_semana + 
##     data
##   Res.Df        RSS Df Sum of Sq      F    Pr(>F)    
## 1    362 2695619426                                  
## 2    361 2615517078  1  80102349 11.056 0.0009747 ***
## ---
## Signif. codes:  0 '***' 0.001 '**' 0.01 '*' 0.05 '.' 0.1 ' ' 1
\end{verbatim}

\hypertarget{adicionar-precipitacao_mm-1}{%
\subsubsection{Adicionar
precipitacao\_mm}\label{adicionar-precipitacao_mm-1}}

\begin{Shaded}
\begin{Highlighting}[]
\NormalTok{modelo\_mais\_precipitacao\_mm }\OtherTok{=} \FunctionTok{lm}\NormalTok{(consumo\_de\_cerveja\_litros }\SpecialCharTok{\textasciitilde{}} 
\NormalTok{                                temperatura\_media\_c}\SpecialCharTok{+}\NormalTok{final\_de\_semana}\SpecialCharTok{+}\NormalTok{precipitacao\_mm)}
\FunctionTok{anova}\NormalTok{(modelo\_reduzido,modelo\_mais\_precipitacao\_mm)}
\end{Highlighting}
\end{Shaded}

\begin{verbatim}
## Analysis of Variance Table
## 
## Model 1: consumo_de_cerveja_litros ~ temperatura_media_c + final_de_semana
## Model 2: consumo_de_cerveja_litros ~ temperatura_media_c + final_de_semana + 
##     precipitacao_mm
##   Res.Df        RSS Df Sum of Sq      F    Pr(>F)    
## 1    362 2695619426                                  
## 2    361 2386755919  1 308863507 46.716 3.508e-11 ***
## ---
## Signif. codes:  0 '***' 0.001 '**' 0.01 '*' 0.05 '.' 0.1 ' ' 1
\end{verbatim}

\hypertarget{conclusuxe3o-5}{%
\subsubsection{Conclusão:}\label{conclusuxe3o-5}}

O Fmax é igual 46.716 sendo precipitacao\_mm o mais provável de ser
adicionado.

\hypertarget{validando-a-variavel-1}{%
\subsubsection{Validando a variavel}\label{validando-a-variavel-1}}

\begin{Shaded}
\begin{Highlighting}[]
\FunctionTok{qf}\NormalTok{(}\FloatTok{0.95}\NormalTok{, }\DecValTok{1}\NormalTok{ , }\DecValTok{361}\NormalTok{)}
\end{Highlighting}
\end{Shaded}

\begin{verbatim}
## [1] 3.867347
\end{verbatim}

\hypertarget{conclusuxe3o-6}{%
\paragraph{Conclusão:}\label{conclusuxe3o-6}}

Fmax(46.716) \textgreater{} Ftab(3.867347) aceitamos o modelo completo
como consumo\_de\_cerveja\_litros \textasciitilde{}
temperatura\_media\_c+final\_de\_semana+precipitacao\_mm

\hypertarget{novo-modelo-reduzido-1}{%
\subsubsection{Novo Modelo Reduzido}\label{novo-modelo-reduzido-1}}

\begin{Shaded}
\begin{Highlighting}[]
\NormalTok{modelo\_reduzido }\OtherTok{=} \FunctionTok{lm}\NormalTok{(consumo\_de\_cerveja\_litros }\SpecialCharTok{\textasciitilde{}} 
\NormalTok{                       temperatura\_media\_c}\SpecialCharTok{+}\NormalTok{final\_de\_semana}\SpecialCharTok{+}\NormalTok{precipitacao\_mm)}
\end{Highlighting}
\end{Shaded}

\hypertarget{adicionar-data-2}{%
\subsubsection{Adicionar data}\label{adicionar-data-2}}

\begin{Shaded}
\begin{Highlighting}[]
\NormalTok{modelo\_mais\_data }\OtherTok{=} \FunctionTok{lm}\NormalTok{(consumo\_de\_cerveja\_litros }\SpecialCharTok{\textasciitilde{}} 
\NormalTok{                        temperatura\_media\_c}\SpecialCharTok{+}\NormalTok{final\_de\_semana}\SpecialCharTok{+}\NormalTok{precipitacao\_mm}\SpecialCharTok{+}\NormalTok{data)}
\FunctionTok{summary}\NormalTok{(}\FunctionTok{aov}\NormalTok{(modelo\_mais\_data))}
\end{Highlighting}
\end{Shaded}

\begin{verbatim}
##                      Df    Sum Sq   Mean Sq F value   Pr(>F)    
## temperatura_media_c   1 2.326e+09 2.326e+09  363.50  < 2e-16 ***
## final_de_semana       1 2.023e+09 2.023e+09  316.13  < 2e-16 ***
## precipitacao_mm       1 3.089e+08 3.089e+08   48.27 1.75e-11 ***
## data                  1 8.328e+07 8.328e+07   13.02 0.000353 ***
## Residuals           360 2.303e+09 6.399e+06                     
## ---
## Signif. codes:  0 '***' 0.001 '**' 0.01 '*' 0.05 '.' 0.1 ' ' 1
\end{verbatim}

\begin{Shaded}
\begin{Highlighting}[]
\FunctionTok{anova}\NormalTok{(modelo\_reduzido,modelo\_mais\_data)}
\end{Highlighting}
\end{Shaded}

\begin{verbatim}
## Analysis of Variance Table
## 
## Model 1: consumo_de_cerveja_litros ~ temperatura_media_c + final_de_semana + 
##     precipitacao_mm
## Model 2: consumo_de_cerveja_litros ~ temperatura_media_c + final_de_semana + 
##     precipitacao_mm + data
##   Res.Df        RSS Df Sum of Sq      F    Pr(>F)    
## 1    361 2386755919                                  
## 2    360 2303474722  1  83281197 13.016 0.0003526 ***
## ---
## Signif. codes:  0 '***' 0.001 '**' 0.01 '*' 0.05 '.' 0.1 ' ' 1
\end{verbatim}

\hypertarget{conclusuxe3o-7}{%
\subsubsection{Conclusão:}\label{conclusuxe3o-7}}

O Fmax é igual 13.016 sendo data provável de ser adicionado.

\hypertarget{validando-a-variavel-2}{%
\subsubsection{Validando a variavel}\label{validando-a-variavel-2}}

\begin{Shaded}
\begin{Highlighting}[]
\FunctionTok{qf}\NormalTok{(}\FloatTok{0.95}\NormalTok{, }\DecValTok{1}\NormalTok{ , }\DecValTok{360}\NormalTok{)}
\end{Highlighting}
\end{Shaded}

\begin{verbatim}
## [1] 3.867419
\end{verbatim}

\hypertarget{conclusuxe3o-8}{%
\paragraph{Conclusão:}\label{conclusuxe3o-8}}

Fmax(13.016) \textgreater{} Ftab(3.867419) aceitamos o modelo completo
como consumo\_de\_cerveja\_litros \textasciitilde{}
temperatura\_media\_c+final\_de\_semana+precipitacao\_mm+data

\hypertarget{conclusuxe3o-9}{%
\subsubsection{Conclusão:}\label{conclusuxe3o-9}}

O Modelo da questão D e E chegaram no mesmo resultado.

\hypertarget{questuxe3o-f}{%
\subsection{Questão F}\label{questuxe3o-f}}

Escolha um dos modelos ajustados em (d) ou (e) e faça a análise completa
dos resíduos do modelo, verificando todas as pressuposições do modelo.
Apresente os gráficos dos resíduos padronizados contra: y estimado,
variáveis independentes, ordem das observações. Apresente todas as
conclusões. Complemente as conclusões com os testes de Shapiro Wilk,
Durbin Watson. Discuta sobre a necessidade de transformação na variável
resposta, ou de usar mínimos quadrados ponderados, justificando.

\begin{Shaded}
\begin{Highlighting}[]
\NormalTok{residuos }\OtherTok{\textless{}{-}} \FunctionTok{residuals}\NormalTok{(modelo\_completo)}
\end{Highlighting}
\end{Shaded}

\hypertarget{gruxe1fico-de-dispersuxe3o-dos-resuxedduos-vs.-valores-ajustados}{%
\subsubsection{Gráfico de Dispersão dos Resíduos vs.~Valores
Ajustados}\label{gruxe1fico-de-dispersuxe3o-dos-resuxedduos-vs.-valores-ajustados}}

\begin{Shaded}
\begin{Highlighting}[]
\FunctionTok{plot}\NormalTok{(modelo\_completo}\SpecialCharTok{$}\NormalTok{fitted.values, residuos)}
\end{Highlighting}
\end{Shaded}

\includegraphics{Ultimo_Trabalho_Regressao_files/figure-latex/unnamed-chunk-34-1.pdf}

\hypertarget{conclusuxf5es}{%
\paragraph{Conclusões:}\label{conclusuxf5es}}

Os pontos distribuem-se de forma aleatória, indicando assim
homocedasticidade e homogênea, ou seja, a variabilidade dos resíduos é
constante em vários níveis de predição. Se houvesse padrões, como um
aumento nos resíduos à medida que os valores ajustados aumentam, isso
sugeriria a presença de heterocedasticidade. Tal cenário poderia indicar
que o residuo do modelo não possui variabilidade costante.

(Solução: fazer transformação em Y ou usar Mínimos Quadrados
Ponderados.)

A homocedasticidade é provavelmente violada se\ldots{}

\begin{enumerate}
\def\labelenumi{\arabic{enumi}.}
\tightlist
\item
  Se os resíduos aumentam ou diminuem com os valores ajustados.
\item
  Se os pontos formam uma curva ao redor de zero e não estão dispostos
  aleatoriamente.
\item
  Poucos pontos no gráfico ficam muito distantes dos demais.
\end{enumerate}

\hypertarget{gruxe1fico-de-dispersuxe3o-dos-resuxedduos-vs.-data}{%
\subsubsection{Gráfico de Dispersão dos Resíduos
vs.~Data}\label{gruxe1fico-de-dispersuxe3o-dos-resuxedduos-vs.-data}}

\begin{Shaded}
\begin{Highlighting}[]
\FunctionTok{plot}\NormalTok{(data, residuos)}
\end{Highlighting}
\end{Shaded}

\includegraphics{Ultimo_Trabalho_Regressao_files/figure-latex/unnamed-chunk-35-1.pdf}

\hypertarget{conclusuxf5es-1}{%
\paragraph{Conclusões:}\label{conclusuxf5es-1}}

\begin{enumerate}
\def\labelenumi{\arabic{enumi}.}
\tightlist
\item
  Os resíduos se distribuem aleatoriamente em torno de zero.
\item
  Não se observa nenhum padrão.
\item
  Isso indica que: a variância é constante; e a relação entre as
  variáveis é linear.
\end{enumerate}

\hypertarget{gruxe1fico-de-dispersuxe3o-dos-resuxedduos-vs.-temperatura_media_c}{%
\subsubsection{Gráfico de Dispersão dos Resíduos
vs.~Temperatura\_media\_c}\label{gruxe1fico-de-dispersuxe3o-dos-resuxedduos-vs.-temperatura_media_c}}

\begin{Shaded}
\begin{Highlighting}[]
\FunctionTok{plot}\NormalTok{(temperatura\_media\_c, residuos)}
\end{Highlighting}
\end{Shaded}

\includegraphics{Ultimo_Trabalho_Regressao_files/figure-latex/unnamed-chunk-36-1.pdf}

\hypertarget{conclusuxf5es-2}{%
\paragraph{Conclusões:}\label{conclusuxf5es-2}}

Tem uma leve diferença entre os valores de 15 a 20 e depois tem uma
aumento, pode repesentar uma variância não constante, mas é necessario
realizar mais testes.

\begin{enumerate}
\def\labelenumi{\arabic{enumi}.}
\tightlist
\item
  Os resíduos se distribuem aleatoriamente em torno de zero.
\item
  Não se observa nenhum padrão.
\item
  Isso indica que: a variância é constante; e a relação entre as
  variáveis é linear.
\end{enumerate}

\hypertarget{gruxe1fico-de-dispersuxe3o-dos-resuxedduos-vs.-precipitacao_mm}{%
\subsubsection{Gráfico de Dispersão dos Resíduos
vs.~precipitacao\_mm}\label{gruxe1fico-de-dispersuxe3o-dos-resuxedduos-vs.-precipitacao_mm}}

\begin{Shaded}
\begin{Highlighting}[]
\FunctionTok{plot}\NormalTok{(precipitacao\_mm , residuos)}
\end{Highlighting}
\end{Shaded}

\includegraphics{Ultimo_Trabalho_Regressao_files/figure-latex/unnamed-chunk-37-1.pdf}

Conclusões:

Possui uma concentraçao no zero e tem uma leve diferença entre os
valores de 0 a 20, em seguida tem uma diminuição, pode repesentar uma
variância não constante, mas é necessario realizar mais testes.

\begin{enumerate}
\def\labelenumi{\arabic{enumi}.}
\tightlist
\item
  Os resíduos se distribuem aleatoriamente em torno de zero.
\item
  Não se observa nenhum padrão.
\item
  Isso indica que: a variância é constante; e a relação entre as
  variáveis é linear.
\end{enumerate}

\hypertarget{gruxe1fico-de-dispersuxe3o-dos-resuxedduos-vs.-final_de_semana}{%
\subsubsection{Gráfico de Dispersão dos Resíduos
vs.~final\_de\_semana}\label{gruxe1fico-de-dispersuxe3o-dos-resuxedduos-vs.-final_de_semana}}

\begin{Shaded}
\begin{Highlighting}[]
\FunctionTok{plot}\NormalTok{(final\_de\_semana , residuos)}
\end{Highlighting}
\end{Shaded}

\includegraphics{Ultimo_Trabalho_Regressao_files/figure-latex/unnamed-chunk-38-1.pdf}

\hypertarget{conclusuxf5es-3}{%
\paragraph{Conclusões:}\label{conclusuxf5es-3}}

\begin{enumerate}
\def\labelenumi{\arabic{enumi}.}
\tightlist
\item
  Os resíduos se distribuem aleatoriamente em torno de zero.
\item
  Não se observa nenhum padrão.
\item
  Isso indica que: a variância é constante; e a relação entre as
  variáveis é linear.
\end{enumerate}

\hypertarget{gruxe1fico-de-dispersuxe3o-dos-resuxedduos-vs.-ordem-das-observauxe7uxf5es}{%
\section{Gráfico de dispersão dos resíduos vs.~ordem das
observações}\label{gruxe1fico-de-dispersuxe3o-dos-resuxedduos-vs.-ordem-das-observauxe7uxf5es}}

\begin{Shaded}
\begin{Highlighting}[]
\FunctionTok{plot}\NormalTok{(}\DecValTok{1}\SpecialCharTok{:}\FunctionTok{length}\NormalTok{(residuos), residuos)}
\end{Highlighting}
\end{Shaded}

\includegraphics{Ultimo_Trabalho_Regressao_files/figure-latex/unnamed-chunk-39-1.pdf}

A dispersão aleatória dos resíduos ao longo da ordem das observações
mostra que o modelo está bem ajustado aos dados. Não há indicações de
que o modelo esteja cometendo erros de maneira sistemática ou
previsível.

\hypertarget{gruxe1fico-qq-plots}{%
\subsubsection{Gráfico QQ-plots}\label{gruxe1fico-qq-plots}}

\begin{Shaded}
\begin{Highlighting}[]
\FunctionTok{qqnorm}\NormalTok{(residuos)}
\FunctionTok{qqline}\NormalTok{(residuos) }
\end{Highlighting}
\end{Shaded}

\includegraphics{Ultimo_Trabalho_Regressao_files/figure-latex/unnamed-chunk-40-1.pdf}

\hypertarget{conclusuxf5es-4}{%
\paragraph{Conclusões:}\label{conclusuxf5es-4}}

Podemos ver que os resíduos tendem a se desviar um pouco da linha perto
das caudas, o que pode indicar que eles não estão normalmente
distribuídos.

\hypertarget{teste-de-normalidade-dos-resuxedduos-shapiro-wilk}{%
\subsubsection{Teste de normalidade dos resíduos
(Shapiro-Wilk)}\label{teste-de-normalidade-dos-resuxedduos-shapiro-wilk}}

\begin{Shaded}
\begin{Highlighting}[]
\FunctionTok{shapiro.test}\NormalTok{(residuos)}
\end{Highlighting}
\end{Shaded}

\begin{verbatim}
## 
##  Shapiro-Wilk normality test
## 
## data:  residuos
## W = 0.98964, p-value = 0.01106
\end{verbatim}

\hypertarget{hipuxf3tese-5}{%
\paragraph{Hipótese:}\label{hipuxf3tese-5}}

\[
\left\{ \begin{array}{rc} 
H0: Os \ erros \ têm \ distribuição \ normal \\ 
H1: Os \ erros \ não \ têm \ distribuição \ normal \\ 
\end{array}\right.
\]

\hypertarget{conclusuxf5es-5}{%
\subparagraph{Conclusões:}\label{conclusuxf5es-5}}

Como resultado, o teste retornará a estatística W, que terá um valor de
significância associada, o valor-p.~Para dizer que uma distribuição é
normal, o valor p precisa ser maior do que 0,05. Logo rejeitar a
hipótese nula concluimos que os erros não tem distribuição normal.

\hypertarget{teste-de-autocorrelauxe7uxe3o-dos-resuxedduos-durbin-watson}{%
\subsubsection{Teste de autocorrelação dos resíduos
(Durbin-Watson)}\label{teste-de-autocorrelauxe7uxe3o-dos-resuxedduos-durbin-watson}}

\begin{Shaded}
\begin{Highlighting}[]
\FunctionTok{durbinWatsonTest}\NormalTok{(modelo\_completo)}
\end{Highlighting}
\end{Shaded}

\begin{verbatim}
##  lag Autocorrelation D-W Statistic p-value
##    1      0.03100611      1.919447   0.366
##  Alternative hypothesis: rho != 0
\end{verbatim}

\hypertarget{hipuxf3tese-6}{%
\paragraph{Hipótese:}\label{hipuxf3tese-6}}

\[
\left\{ \begin{array}{rc} 
H0: 𝜌 = 0 \\ 
H1: 𝜌 \neq 0 \\ 
\end{array}\right.
\]

\hypertarget{conclusuxf5es-6}{%
\subparagraph{Conclusões:}\label{conclusuxf5es-6}}

Ele faz uma estatística que testa se eles são independentes. Aí, o
p-valor que a gente compara é com 0.05, nesse caso, o p-valor não foi
menor do que 0.05. Logo, a gente não tem evidência suficiente para
rejeitar a hipótese nula. Se eu não consigo rejeitar é porque eu não
tenho evidências para dizer que elas não são independentes. Então, eu
aceito que são independentes e não há autocorrelação significativa.

\hypertarget{conclusuxf5es-geral}{%
\subsubsection{Conclusões Geral:}\label{conclusuxf5es-geral}}

É possível que seja necessário transformar a variável de resposta para
abordar a falta de normalidade nos resíduos, facilitando assim a
aprovação no teste de Shapiro-Wilk. No entanto, em relação à aplicação
de mínimos quadrados ponderados, a evidência de falta de
homocedasticidade nos resíduos não é conclusiva, necessitando de mais
testes.

\hypertarget{questuxe3o-g}{%
\subsection{Questão G}\label{questuxe3o-g}}

Realize a análise de diagnóstico do modelo, apresentando os gráficos de
todas as medidas estudadas (DFFITS, DFBETAS, Distância de Cook,
leverage, resíduos estudentizados). Interprete cada uma usando o gráfico
e CALCULE o ponto de corte visto no livro para comparar.

\hypertarget{dffit}{%
\subsubsection{DFFIT}\label{dffit}}

\begin{Shaded}
\begin{Highlighting}[]
\FunctionTok{ols\_plot\_dffits}\NormalTok{(modelo\_completo)}
\end{Highlighting}
\end{Shaded}

\includegraphics{Ultimo_Trabalho_Regressao_files/figure-latex/unnamed-chunk-43-1.pdf}

\hypertarget{conclusuxf5es-7}{%
\paragraph{Conclusões:}\label{conclusuxf5es-7}}

DFFIT - diferença nos ajustes, é usado para identificar pontos de dados
influentes. Ele quantifica o número de desvios padrão que o valor
ajustado muda quando o i-ésimo ponto de dados é omitido.

Pontos ultrapassao a linha podem ser considerados pontos influentes.

\hypertarget{dfbetas}{%
\subsubsection{DFBETAS}\label{dfbetas}}

\begin{Shaded}
\begin{Highlighting}[]
\FunctionTok{ols\_plot\_dfbetas}\NormalTok{(modelo\_completo)}
\end{Highlighting}
\end{Shaded}

\includegraphics{Ultimo_Trabalho_Regressao_files/figure-latex/unnamed-chunk-44-1.pdf}
\includegraphics{Ultimo_Trabalho_Regressao_files/figure-latex/unnamed-chunk-44-2.pdf}

\hypertarget{conclusuxf5es-8}{%
\paragraph{Conclusões:}\label{conclusuxf5es-8}}

O gráfico sugere que vários pontos têm uma influência considerável na
estimativa da inclinação para x. Neste caso, a inspeção visual da
relação linear entre x e y é possível. O painel mostra a influência de
cada observação nas estimativas dos quatro coeficientes de regressão. As
estatísticas são padronizadas para que todos os gráficos possam utilizar
a mesma escala vertical. As linhas horizontais são desenhadas em
±2/sqrt(n). As observações são chamadas de influentes se tiverem uma
estatística DFBETA que exceda esse valor.

\hypertarget{cooks-distance}{%
\subsubsection{Cook's distance}\label{cooks-distance}}

\begin{Shaded}
\begin{Highlighting}[]
\FunctionTok{ols\_plot\_cooksd\_chart}\NormalTok{(modelo\_completo)}
\end{Highlighting}
\end{Shaded}

\includegraphics{Ultimo_Trabalho_Regressao_files/figure-latex/unnamed-chunk-45-1.pdf}

\hypertarget{conclusuxf5es-9}{%
\paragraph{Conclusões:}\label{conclusuxf5es-9}}

Gráfico de barras da distância de Cook para detectar observações que
influenciam fortemente os valores ajustados do modelo. A distância de
Cook foi introduzida pelo estatístico americano R Dennis Cook em 1977. É
usada para identificar pontos de dados influentes. Depende tanto do
resíduo quanto da alavancagem, ou seja, leva em consideração tanto o
valor x quanto o valor y da observação. Passos para calcular a distância
de Cook:

\begin{itemize}
\tightlist
\item
  exclua as observações uma de cada vez.
\item
  reajuste o modelo de regressão no restante (n−1) observações
\item
  examine quanto todos os valores ajustados mudam quando a i-ésima
  observação é excluída.
\end{itemize}

Um ponto de dados com distância Cook grande indica que o ponto de dados
influencia fortemente os valores ajustados.

\hypertarget{studentized-residuals-vs-leverage-plot}{%
\subsubsection{Studentized Residuals vs Leverage
Plot}\label{studentized-residuals-vs-leverage-plot}}

\begin{Shaded}
\begin{Highlighting}[]
\FunctionTok{ols\_plot\_resid\_lev}\NormalTok{(modelo\_completo)}
\end{Highlighting}
\end{Shaded}

\includegraphics{Ultimo_Trabalho_Regressao_files/figure-latex/unnamed-chunk-46-1.pdf}

\hypertarget{conclusuxf5es-10}{%
\paragraph{Conclusões:}\label{conclusuxf5es-10}}

Gráfico para detecção de outliers e/ou observações com alta influencia

\hypertarget{resuxedduos-studentizados}{%
\subsubsection{Resíduos Studentizados}\label{resuxedduos-studentizados}}

\begin{Shaded}
\begin{Highlighting}[]
\FunctionTok{ols\_plot\_resid\_stud}\NormalTok{(modelo\_completo)}
\end{Highlighting}
\end{Shaded}

\includegraphics{Ultimo_Trabalho_Regressao_files/figure-latex/unnamed-chunk-47-1.pdf}

\hypertarget{conclusuxf5es-11}{%
\paragraph{Conclusões:}\label{conclusuxf5es-11}}

Gráfico para detecção de outliers. Resíduos excluídos estudantis (ou
resíduos estudantis externamente) são os resíduos excluídos divididos
por seu desvio padrão estimado. Os resíduos estudantis serão mais
eficazes para detectar observações Y periféricas do que os resíduos
padronizados. Se uma observação tiver um resíduo estudantil externamente
maior que 3 (em valor absoluto), podemos chamá-la de outlier.

\hypertarget{regressuxe3o-loguxedstica}{%
\section{Regressão Logística}\label{regressuxe3o-loguxedstica}}

\hypertarget{questuxe3o-a-1}{%
\subsection{Questão A}\label{questuxe3o-a-1}}

Escreva a equação do modelo ajustado, capaz de descrever o
relacionamento existente entre a ocorrência de sinistro e as variáveis
em estudo.

\begin{Shaded}
\begin{Highlighting}[]
\NormalTok{sinistro }\OtherTok{\textless{}{-}} \FunctionTok{read.delim}\NormalTok{(}\StringTok{"dados/dadosSinistro.txt"}\NormalTok{)}
\end{Highlighting}
\end{Shaded}

\begin{Shaded}
\begin{Highlighting}[]
\NormalTok{modelo}\OtherTok{=}\FunctionTok{glm}\NormalTok{(Sinistro}\SpecialCharTok{\textasciitilde{}}\NormalTok{Idade}\SpecialCharTok{+}\NormalTok{ECivil}\SpecialCharTok{+}\NormalTok{Sexo,}\AttributeTok{data =}\NormalTok{ sinistro,}
           \AttributeTok{family=}\FunctionTok{binomial}\NormalTok{(}\AttributeTok{link=}\StringTok{"logit"}\NormalTok{))}
\FunctionTok{summary}\NormalTok{(modelo)}
\end{Highlighting}
\end{Shaded}

\begin{verbatim}
## 
## Call:
## glm(formula = Sinistro ~ Idade + ECivil + Sexo, family = binomial(link = "logit"), 
##     data = sinistro)
## 
## Deviance Residuals: 
##     Min       1Q   Median       3Q      Max  
## -1.3375  -0.3070  -0.1441   0.2496   2.0702  
## 
## Coefficients:
##             Estimate Std. Error z value Pr(>|z|)   
## (Intercept)  3.78103    1.94470   1.944  0.05186 . 
## Idade       -0.18957    0.07228  -2.623  0.00872 **
## ECivil      -3.62511    1.65160  -2.195  0.02817 * 
## Sexo         3.70268    1.67313   2.213  0.02690 * 
## ---
## Signif. codes:  0 '***' 0.001 '**' 0.01 '*' 0.05 '.' 0.1 ' ' 1
## 
## (Dispersion parameter for binomial family taken to be 1)
## 
##     Null deviance: 42.541  on 35  degrees of freedom
## Residual deviance: 20.990  on 32  degrees of freedom
## AIC: 28.99
## 
## Number of Fisher Scoring iterations: 6
\end{verbatim}

\hypertarget{questuxe3o-b-1}{%
\subsection{Questão B}\label{questuxe3o-b-1}}

Explicar o significado de cada componente do modelo;

\hypertarget{conclusuxf5es-12}{%
\paragraph{Conclusões:}\label{conclusuxf5es-12}}

beta 1 = - 0.19 o log da razão de chances diminui em 0.19 quando
seaumenta uma unidade na variável idade e mantém as demais
variáveisconstantes. A estimativa negativa de Beta1 indica que
aumentando-se avariável idade, diminui-se a chance de sucesso (sinistro)
Beta2 = -3,62: diminuição no log da razão de chances quando se muda
oestado civil de casado para solteiro. Beta3 = 3,70: aumento no log da
razão de chances quando se muda dosexo masculino para o feminino. A
estimativa positiva de Beta3 indica queo sexo codificado como 1, tem
maior chance de sucesso (sinistro) quandocomparado com o estado
codificado como zero.

O modelo apresenta Pr(\textgreater\textbar z\textbar) é menos que o
nível significância de 0.05. Logo os parâmetros são significativos. O
modelo possui um Intercepto positivo o que siginifica que ele tem uma
curva crescente.

O AIC estima a quantidade relativa de informação perdida por um
determinado modelo: quanto menos informações um modelo perde, maior a
qualidade desse modelo e menor a pontuação AIC.

\hypertarget{questuxe3o-c-1}{%
\subsection{Questão C}\label{questuxe3o-c-1}}

Teste a significância do modelo por meio do Teste de razão de
verossimilhança, apresentando as hipóteses e conclusão. Teste a
significância dos parâmetros do modelo e apresente as conclusões, em
termos das variáveis. Use 1\% de significância.

\begin{Shaded}
\begin{Highlighting}[]
\FunctionTok{paste0}\NormalTok{(}\StringTok{\textquotesingle{}TRV = \textquotesingle{}}\NormalTok{,modelo[[}\StringTok{"null.deviance"}\NormalTok{]]}\SpecialCharTok{{-}}\NormalTok{modelo[[}\StringTok{"deviance"}\NormalTok{]],}\StringTok{\textquotesingle{} e \textquotesingle{}}\NormalTok{, }
       \FunctionTok{qchisq}\NormalTok{(}\FloatTok{0.99}\NormalTok{,}\DecValTok{3}\NormalTok{), }\StringTok{\textquotesingle{} há 95\% e 3p\textquotesingle{}}\NormalTok{)}
\end{Highlighting}
\end{Shaded}

\begin{verbatim}
## [1] "TRV = 21.5506493356017 e 11.3448667301444 há 95% e 3p"
\end{verbatim}

\hypertarget{conclusuxf5es-13}{%
\paragraph{Conclusões:}\label{conclusuxf5es-13}}

Dado que o TRV é maior que qui quadrado há 99\% e com 3 parametros o
modelo como um todo é significativo, rejeitamos h0.

Lembrando:

Considere dois modelos um ``completo'' (Mc ) e o outro ``restrito'' (Mr)
quanto à quantidade de parâmetros.Então a estatística de teste é:

TRV = 2 {[}log(L(Mc )) − log(L(Mr)){]}.

\[
\left\{ \begin{array}{rc} 
H0: Igualdade \ entre \ os \ dois \ modelos \\ 
H1: Difernça \ entre \ os \ dois \ modelos \\ 
\end{array}\right.
\]

Sob a hipóotese de igualdade entre os dois modelos, TRV tem distribuição
aproximada de Qui-Quadrado com graus de liberdade igual ao número de
restrições (diferença entre as quantidades de parâmetros).

\hypertarget{questuxe3o-d-1}{%
\subsection{Questão D}\label{questuxe3o-d-1}}

Interprete as razões de chance e os intervalos de confiança obtidas para
todos os parâmetros;

\hypertarget{intervalos-de-confianuxe7a}{%
\subsubsection{Intervalos de
confiança}\label{intervalos-de-confianuxe7a}}

\begin{Shaded}
\begin{Highlighting}[]
\NormalTok{ICbeta}\OtherTok{=}\FunctionTok{confint.default}\NormalTok{(modelo,}\AttributeTok{level=}\FloatTok{0.95}\NormalTok{)}
\NormalTok{ICbeta}
\end{Highlighting}
\end{Shaded}

\begin{verbatim}
##                   2.5 %      97.5 %
## (Intercept) -0.03051846  7.59258396
## Idade       -0.33123689 -0.04790661
## ECivil      -6.86218543 -0.38803500
## Sexo         0.42339383  6.98195617
\end{verbatim}

\hypertarget{conclusuxf5es-14}{%
\paragraph{Conclusões:}\label{conclusuxf5es-14}}

Isso significa que você tem 95\% de confiança de que a verdadeiro valor
de beta/intercepto esteja entre os valores acima. O valor zero pertence
ao intervalo, logo os parâmetros de beta sao estatisticamente igual a
zero.

\hypertarget{odds-ratio-or-chance}{%
\subsubsection{odds ratio -- OR (chance)}\label{odds-ratio-or-chance}}

\begin{Shaded}
\begin{Highlighting}[]
\NormalTok{OR}\OtherTok{=}\FunctionTok{exp}\NormalTok{(modelo}\SpecialCharTok{$}\NormalTok{coefficients)}
\NormalTok{OR}
\end{Highlighting}
\end{Shaded}

\begin{verbatim}
## (Intercept)       Idade      ECivil        Sexo 
## 43.86131610  0.82731335  0.02664616 40.55564586
\end{verbatim}

\hypertarget{conclusuxf5es-15}{%
\paragraph{Conclusões:}\label{conclusuxf5es-15}}

A chance de haver sinistro(1) é de cerca de 0.83 vezes menos para cada
ano de acrescentado na idade, ou seja, as chances de a haver sinistro é
de 17\% menor a cada ano de acrescentado na idade; A chance de haver
sinistro(1) é de cerca de 0.027 vezes menos quando estado civil igual a
Solteiro(1), ou seja, as chances de a haver sinistro é de 97.3\% menor
se for Solteiro; A chance de haver sinistro(1) é de cerca de 40.55 vezes
maior se o sexo for Feminino(1).

Odds 1 = 0.83 representa a diminuição na chance de sinistro quando
seaumenta uma unidade na idade. A chance de sinistro diminui cerca de17
\% (1-0,83) quando se aumenta um ano na idade e mantém as
demaisvariáveis constantes. Odds2 = 0.027 representa a diminuição na
chance de sinistro de umindividuo solteiro, em comparação com um casado.
Portanto, o indivíduosolteiro tem cerca de 98\% (1-0,027) menos chances
de sinistro do que umcasado Odds3 = 40,55 representa o aumento na chance
de sinistro quando oindivíduo é feminino. Assim, o feminino tem 40 vezes
mais chance desinistro do que o masculino

\hypertarget{intervalos-de-confianuxe7a-para-as-razuxf5es-de-chance}{%
\subsubsection{Intervalos de confiança para as razões de
chance}\label{intervalos-de-confianuxe7a-para-as-razuxf5es-de-chance}}

\begin{Shaded}
\begin{Highlighting}[]
\NormalTok{ICOR}\OtherTok{=}\FunctionTok{exp}\NormalTok{(ICbeta);}
\FunctionTok{round}\NormalTok{((}\FunctionTok{cbind}\NormalTok{(OR, ICOR)),}\DecValTok{3}\NormalTok{)}
\end{Highlighting}
\end{Shaded}

\begin{verbatim}
##                 OR 2.5 %   97.5 %
## (Intercept) 43.861 0.970 1983.432
## Idade        0.827 0.718    0.953
## ECivil       0.027 0.001    0.678
## Sexo        40.556 1.527 1077.023
\end{verbatim}

\hypertarget{conclusuxf5es-16}{%
\paragraph{Conclusões:}\label{conclusuxf5es-16}}

Isso significa que você tem 95\% de confiança de que a verdadeiro valor
de odds ratio esteja entre os valores acima. Já que 1 não pertence ao
intevalo, logo a chance de sucesso para os dois grupos sob comparação é
diferente.

Quando o intervalo de confiança da razão de chances inclui o valor 1,
significa que estatisticamente a razão de chances vale 1, ou seja, a
chance de sucesso para os dois grupos sob comparação é a mesma. Se o IC
para OR inclui o 1, no caso do sexo, significa que a chance de sucesso
para masculino é a mesma que para o feminino.

\hypertarget{questuxe3o-e-1}{%
\subsection{Questão E}\label{questuxe3o-e-1}}

Estime a probabilidade de sinistro associada a um cliente do sexo
masculino, casado com 25 anos de idade

\begin{Shaded}
\begin{Highlighting}[]
\NormalTok{p\_sexo }\OtherTok{=} \DecValTok{0}
\NormalTok{p\_ec }\OtherTok{=} \DecValTok{0}
\NormalTok{p\_idade }\OtherTok{=} \DecValTok{25}

\NormalTok{razaopiest }\OtherTok{=} \FunctionTok{exp}\NormalTok{(modelo[[}\StringTok{"coefficients"}\NormalTok{]][}\StringTok{\textquotesingle{}(Intercept)\textquotesingle{}}\NormalTok{][[}\DecValTok{1}\NormalTok{]] }\SpecialCharTok{+} 
\NormalTok{                 (modelo[[}\StringTok{"coefficients"}\NormalTok{]][}\StringTok{\textquotesingle{}Idade\textquotesingle{}}\NormalTok{][[}\DecValTok{1}\NormalTok{]] }\SpecialCharTok{*}\NormalTok{ p\_idade) }\SpecialCharTok{+}
\NormalTok{                 (modelo[[}\StringTok{"coefficients"}\NormalTok{]][}\StringTok{\textquotesingle{}ECivil\textquotesingle{}}\NormalTok{][[}\DecValTok{1}\NormalTok{]] }\SpecialCharTok{*}\NormalTok{ p\_ec) }\SpecialCharTok{+}
\NormalTok{                 (modelo[[}\StringTok{"coefficients"}\NormalTok{]][}\StringTok{\textquotesingle{}Sexo\textquotesingle{}}\NormalTok{][[}\DecValTok{1}\NormalTok{]] }\SpecialCharTok{*}\NormalTok{ p\_sexo)}
\NormalTok{                 )}

\FunctionTok{paste0}\NormalTok{(}\StringTok{\textquotesingle{}Estimação da probabilidade de sinistro = \textquotesingle{}}\NormalTok{,}
\NormalTok{       (razaopiest }\SpecialCharTok{/}\NormalTok{(}\DecValTok{1} \SpecialCharTok{+}\NormalTok{ razaopiest)))}
\end{Highlighting}
\end{Shaded}

\begin{verbatim}
## [1] "Estimação da probabilidade de sinistro = 0.277226495853509"
\end{verbatim}

\hypertarget{questuxe3o-f-1}{%
\subsection{Questão F}\label{questuxe3o-f-1}}

Para o mesmo cliente citado no item anterior, qual a probabilidade de
sinistro se ele for solteiro?

\begin{Shaded}
\begin{Highlighting}[]
\NormalTok{p\_sexo }\OtherTok{=} \DecValTok{0}
\NormalTok{p\_ec }\OtherTok{=} \DecValTok{1}
\NormalTok{p\_idade }\OtherTok{=} \DecValTok{25}

\NormalTok{razaopiest }\OtherTok{=} \FunctionTok{exp}\NormalTok{(modelo[[}\StringTok{"coefficients"}\NormalTok{]][}\StringTok{\textquotesingle{}(Intercept)\textquotesingle{}}\NormalTok{][[}\DecValTok{1}\NormalTok{]] }\SpecialCharTok{+} 
\NormalTok{                 (modelo[[}\StringTok{"coefficients"}\NormalTok{]][}\StringTok{\textquotesingle{}Idade\textquotesingle{}}\NormalTok{][[}\DecValTok{1}\NormalTok{]] }\SpecialCharTok{*}\NormalTok{ p\_idade) }\SpecialCharTok{+}
\NormalTok{                 (modelo[[}\StringTok{"coefficients"}\NormalTok{]][}\StringTok{\textquotesingle{}ECivil\textquotesingle{}}\NormalTok{][[}\DecValTok{1}\NormalTok{]] }\SpecialCharTok{*}\NormalTok{ p\_ec) }\SpecialCharTok{+}
\NormalTok{                 (modelo[[}\StringTok{"coefficients"}\NormalTok{]][}\StringTok{\textquotesingle{}Sexo\textquotesingle{}}\NormalTok{][[}\DecValTok{1}\NormalTok{]] }\SpecialCharTok{*}\NormalTok{ p\_sexo)}
\NormalTok{                 )}

\FunctionTok{paste0}\NormalTok{(}\StringTok{\textquotesingle{}Estimação da probabilidade de sinistro = \textquotesingle{}}\NormalTok{,}
\NormalTok{       (razaopiest }\SpecialCharTok{/}\NormalTok{(}\DecValTok{1} \SpecialCharTok{+}\NormalTok{ razaopiest)))}
\end{Highlighting}
\end{Shaded}

\begin{verbatim}
## [1] "Estimação da probabilidade de sinistro = 0.010116983059195"
\end{verbatim}

\hypertarget{questuxe3o-g-1}{%
\subsection{Questão G}\label{questuxe3o-g-1}}

Compare os resultados obtidos nos dois itens anteriores e reflita sobre
as estratégias que poderiam ser adotadas pela companhia para atrair
novos clientes.

\hypertarget{conclusuxf5es-17}{%
\paragraph{Conclusões:}\label{conclusuxf5es-17}}

Com base na estimação anterioir e para os parametros colocados a prova
podemos dizer que para a companhia seria benefico focar em atrai
solteiros dados que a chance de haver sinistro diminui muito.

\hypertarget{questuxe3o-h}{%
\subsection{Questão H}\label{questuxe3o-h}}

Considere um nível de significância de 1\% para excluir variáveis do
modelo e comparar o modelo reduzido com o modelo completo, usando o
teste de razão de verossimilhança. Decida qual o melhor modelo, com base
nesse teste.

\begin{Shaded}
\begin{Highlighting}[]
\NormalTok{modelo}\OtherTok{=}\FunctionTok{glm}\NormalTok{(Sinistro}\SpecialCharTok{\textasciitilde{}}\NormalTok{Idade}\SpecialCharTok{+}\NormalTok{ECivil}\SpecialCharTok{+}\NormalTok{Sexo,}\AttributeTok{data =}\NormalTok{ sinistro,}
           \AttributeTok{family=}\FunctionTok{binomial}\NormalTok{(}\AttributeTok{link=}\StringTok{"logit"}\NormalTok{))}
\FunctionTok{summary}\NormalTok{(modelo)}
\end{Highlighting}
\end{Shaded}

\begin{verbatim}
## 
## Call:
## glm(formula = Sinistro ~ Idade + ECivil + Sexo, family = binomial(link = "logit"), 
##     data = sinistro)
## 
## Deviance Residuals: 
##     Min       1Q   Median       3Q      Max  
## -1.3375  -0.3070  -0.1441   0.2496   2.0702  
## 
## Coefficients:
##             Estimate Std. Error z value Pr(>|z|)   
## (Intercept)  3.78103    1.94470   1.944  0.05186 . 
## Idade       -0.18957    0.07228  -2.623  0.00872 **
## ECivil      -3.62511    1.65160  -2.195  0.02817 * 
## Sexo         3.70268    1.67313   2.213  0.02690 * 
## ---
## Signif. codes:  0 '***' 0.001 '**' 0.01 '*' 0.05 '.' 0.1 ' ' 1
## 
## (Dispersion parameter for binomial family taken to be 1)
## 
##     Null deviance: 42.541  on 35  degrees of freedom
## Residual deviance: 20.990  on 32  degrees of freedom
## AIC: 28.99
## 
## Number of Fisher Scoring iterations: 6
\end{verbatim}

\begin{Shaded}
\begin{Highlighting}[]
\NormalTok{deviance\_completo }\OtherTok{=}\NormalTok{ modelo[[}\StringTok{"deviance"}\NormalTok{]]}

\NormalTok{modelo}\OtherTok{=}\FunctionTok{glm}\NormalTok{(Sinistro}\SpecialCharTok{\textasciitilde{}}\NormalTok{Idade,}\AttributeTok{data =}\NormalTok{ sinistro,}
           \AttributeTok{family=}\FunctionTok{binomial}\NormalTok{(}\AttributeTok{link=}\StringTok{"logit"}\NormalTok{))}
\FunctionTok{summary}\NormalTok{(modelo)}
\end{Highlighting}
\end{Shaded}

\begin{verbatim}
## 
## Call:
## glm(formula = Sinistro ~ Idade, family = binomial(link = "logit"), 
##     data = sinistro)
## 
## Deviance Residuals: 
##     Min       1Q   Median       3Q      Max  
## -1.2066  -0.9398  -0.4521   1.2628   1.6573  
## 
## Coefficients:
##             Estimate Std. Error z value Pr(>|z|)  
## (Intercept)  1.54670    1.26218   1.225   0.2204  
## Idade       -0.08212    0.04364  -1.882   0.0599 .
## ---
## Signif. codes:  0 '***' 0.001 '**' 0.01 '*' 0.05 '.' 0.1 ' ' 1
## 
## (Dispersion parameter for binomial family taken to be 1)
## 
##     Null deviance: 42.541  on 35  degrees of freedom
## Residual deviance: 37.033  on 34  degrees of freedom
## AIC: 41.033
## 
## Number of Fisher Scoring iterations: 5
\end{verbatim}

\begin{Shaded}
\begin{Highlighting}[]
\NormalTok{deviance\_idade }\OtherTok{=}\NormalTok{ modelo[[}\StringTok{"deviance"}\NormalTok{]]}

\FunctionTok{paste0}\NormalTok{(}\StringTok{\textquotesingle{}TRV = \textquotesingle{}}\NormalTok{,deviance\_idade}\SpecialCharTok{{-}}\NormalTok{deviance\_completo,}\StringTok{\textquotesingle{} e \textquotesingle{}}\NormalTok{, }
       \FunctionTok{qchisq}\NormalTok{(}\FloatTok{0.99}\NormalTok{,}\DecValTok{2}\NormalTok{), }\StringTok{\textquotesingle{} há 99\% e 4{-}2=2p\textquotesingle{}}\NormalTok{)}
\end{Highlighting}
\end{Shaded}

\begin{verbatim}
## [1] "TRV = 16.0425922111344 e 9.21034037197618 há 99% e 4-2=2p"
\end{verbatim}

\hypertarget{conclusuxf5es-18}{%
\paragraph{Conclusões:}\label{conclusuxf5es-18}}

Dado que o TRV é maior que qui quadrado há 99\% e com 2 parametros,
temos que o modelo com ECivil+Sexo(completo) como um todo é
significativo. Rejeitamos h0.

Lembrando:

\[
\left\{ \begin{array}{lc} 
H0:  Modelo \ só \ com \ 1 \ variável (X1) \ é \ melhor \ que \ o \ modelo \ com \ 3 \\
\ \ \ \ \ \ \ \ \ variáveis(ou \ seja, \ a \ contribuição \ de \ X2 \ e \ X3 \ não \ é \ significativa) \\ 
H1:  Modelo \ completo \ é \ melhor \ que \ o \ reduzido \\ 
\end{array}\right.
\]

Sob a hipóotese de igualdade entre os dois modelos, TRV tem distribuição
aproximada de Qui-Quadrado com graus de liberdade igual ao número de
restrições (diferença entre as quantidades de parâmetros).

Quanto mais elevada a diferença entre os dois valores, maioro potencial
dos coeficientes para estimar probabilidadesassociadas a ocorrência de
determinado evento.

Essa diferença serve para testar a hipótese de que todos oscoeficientes
do modelo logístico são iguais a zero, tal como severifica na
distribuição F.

\end{document}
